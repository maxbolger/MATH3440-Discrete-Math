\documentclass{article}
\usepackage[utf8]{inputenc}
\usepackage{graphicx}
\usepackage{listings}
\usepackage[ruled, lined, linesnumbered, commentsnumbered, longend]{algorithm2e}
\usepackage{xcolor}
\newtheorem{definition}{Definition}

%New colors defined below
\definecolor{codegreen}{rgb}{0,0.6,0}
\definecolor{codegray}{rgb}{0.5,0.5,0.5}
\definecolor{codepurple}{rgb}{0.58,0,0.82}
\definecolor{backcolour}{rgb}{0.95,0.95,0.92}

\lstdefinestyle{mystyle}{
  backgroundcolor=\color{backcolour}, commentstyle=\color{codegreen},
  keywordstyle=\color{magenta},
  numberstyle=\tiny\color{codegray},
  stringstyle=\color{codepurple},
  basicstyle=\ttfamily\footnotesize,
  breakatwhitespace=false,         
  breaklines=true,                 
  captionpos=b,                    
  keepspaces=true,                 
  numbers=left,                    
  numbersep=5pt,                  
  showspaces=false,                
  showstringspaces=false,
  showtabs=false,                  
  tabsize=2
}

\lstset{style=mystyle}

\title{Homework 21 - MATH 3440-1}
\author{Max Bolger}
\date{November 2021}

\begin{document}

\maketitle

\section{Introduction} This is a LaTeX writing assignment focusing on proofs and \textit{mathematical induction}. It was assigned by Professor Ken Takata and is due on 11/15/21.

\begin{definition}
Mathematical induction is a mathematical proof technique. It is essentially used to prove that a statement $P(n)$ holds for every natural number $n = 0, 1, 2, 3, ...$ (to be further explained in section 3.1).
\end{definition}

\section{Write up in LaTeX a proof regarding the sum $\sum_{i=1}^{n} i = 1 + 2 + 3 + 4 + ... + n$ and find a closed form that requires at most $O(1)$ arithmetic operations (constant time).}

\begin{definition}
An algorithm is said to be O(1) (or constant time) if the value of T(n) is bounded by a value that does not depend on the size of the input. Examples of this may include accessing the index of an item in a list or array.
\end{definition}

\subsection{Closed Form}
A closed form of this equation that requires at most $O(1)$ arithmetic operations is:

\begin{equation}
\frac{n(n+1)}{2}
\end{equation}

\subsection{Proof}
Let us assign the following equation to $S$.

\begin{equation}
1 + 2 + 3 + ... + n = S
\end{equation}
\newline
and add the following equation to it (this equation is the previous equation reversed, starting with $n$ and adding $(n-1)$ until we reach $1$).

\begin{equation}
n + (n-1) + (n-2) + ... + 1 = S
\end{equation}
\newline
This step can be better visualized if we stack the components of the addition
\begin{center}
\begin{tabular}{ccccccc}
  & 1 +&2 +& 3 +& ... & + n = S\\
+ & n +&(n-1) +&(n-2) +& ... & + 1 = S\\
\hline
  & (n+1) +&(n+1) +&(n+1) +& ... & + (n+1) = 2S\\
\end{tabular}
\end{center}
\newline
\newline
\newline
We then get a sum of 


\begin{equation}
(n+1) + (n+1) + (n+1) ... + (n+1) = 2S
\end{equation}
\newline
This is equivalent to $n$ amounts of $(n+1)$, so we can rewrite this as

\begin{equation}
n(n+1) = 2S
\end{equation}
\newline
then divide both sides by 2 to isolate $S$

\begin{equation}
\frac{n(n+1)}{2} = S
\end{equation}
\newline
and we arrive at our closed form.


\section{Now prove your formula using mathematical induction.}

\subsection{Intro}
The principle of mathematical induction is to prove that $P(n)$ is true for all positive integers, $n$, where $P(n)$ is a propositional function. Two steps must be completed in mathematical induction:
\newline
\newline
- \textbf{Basis Step (or atomic step, base/atomic case):} Verify that $P(1)$ is true.
\newline
\newline
- \textbf{Inductive Step (or inductive case):} Show that the conditional statement $P(k) \rightarrow P(k+1)$ is true for all positive integers $k$.

\subsection{Base/Atomic Case}
In section 2 we proved that the closed form for the following

\begin{equation}
\sum_{i=1}^{n} i = 1 + 2 + 3 + 4 + ... + n
\end{equation}
\newline
is equivalent to
\begin{equation}
\frac{n(n+1)}{2}
\end{equation}
\newline
This can be written as 
\begin{equation}
\sum_{i=1}^{n} i = \frac{n(n+1)}{2}
\end{equation}
\newline
As 3.1 states, the base case must verify that $P(1)$ is true.
\newline
\newline
\begin{equation}
P(n) \equiv True \textrm{  means } \sum_{i=1}^{n} i = \frac{n(n+1)}{2} = 1
\end{equation}
\newline
\newline
\begin{equation}
P(1) \textrm{  means } \sum_{i=1}^{1} i = \frac{1(2)}{2} = 1
\end{equation}
\newline
\newline
This is true, and our base case is satisfied.

\subsection{Inductive Case}
Now we must show that the conditional statement $P(k) \rightarrow P(k+1)$ is true for all positive integers $k$.
\newline
\newline
\begin{equation}
\textrm{Assume  } P(k)\textrm{:   } \sum_{i=1}^{k} i = \frac{k(k+1)}{2}
\end{equation}
\newline
\newline
\begin{equation}
\textrm{Show  } P(k+1)\textrm{:   } \sum_{i=1}^{k+1} i = \frac{(k+1)[(k+1)+1]}{2}
\end{equation}
\newline
\newline
$P(1)$ was true, now we must test the inductive case with $P(k+1)$. We will test $P(1+1)$
\begin{equation}
\textrm{Show  } P(1+1)\textrm{:   } \sum_{i=1}^{1+1} i = \frac{(1+1)[(1+1)+1]}{2} = 3
\end{equation}
\newline
\newline
This proves that closed form for $P(1) \rightarrow P(1+1)$. We can programmatically test this mathematical induction by calculating the equation for more integers to satisfy $P(k+1)$ for $k$ integers.



\section{Bonus - Code}

\begin{lstlisting}[language=Python, caption = This script tests if the equation and closed form are equal for a range of a given number. A boolean for each iteration is passed to an list. If all booleans in the list are true we know our test has passed.]
def eqn(n):
  sum = 0
  for i in range(1,n+1):
    sum+=i
  return sum

def clsd_frm(n):
  return (n*(n+1)) / 2

def test(n):
  bool_array = []
  for i in range(1,n+1):
    bool_array.append(eqn(i) == clsd_frm(i))
  return all(bool_array)

>>> test(10000)
True

 
\end{lstlisting}



\end{document}
