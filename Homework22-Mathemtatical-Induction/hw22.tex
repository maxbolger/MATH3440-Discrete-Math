\documentclass{article}
\usepackage[utf8]{inputenc}
\usepackage{graphicx}
\usepackage{listings}
\usepackage[ruled, lined, linesnumbered, commentsnumbered, longend]{algorithm2e}
\usepackage{xcolor}
\newtheorem{definition}{Definition}

%New colors defined below
\definecolor{codegreen}{rgb}{0,0.6,0}
\definecolor{codegray}{rgb}{0.5,0.5,0.5}
\definecolor{codepurple}{rgb}{0.58,0,0.82}
\definecolor{backcolour}{rgb}{0.95,0.95,0.92}

\lstdefinestyle{mystyle}{
  backgroundcolor=\color{backcolour}, commentstyle=\color{codegreen},
  keywordstyle=\color{magenta},
  numberstyle=\tiny\color{codegray},
  stringstyle=\color{codepurple},
  basicstyle=\ttfamily\footnotesize,
  breakatwhitespace=false,         
  breaklines=true,                 
  captionpos=b,                    
  keepspaces=true,                 
  numbers=left,                    
  numbersep=5pt,                  
  showspaces=false,                
  showstringspaces=false,
  showtabs=false,                  
  tabsize=2
}

\lstset{style=mystyle}

\title{Homework 22 - MATH 3440-1}
\author{Max Bolger}
\date{November 2021}

\begin{document}

\maketitle

\section{Introduction} This is a LaTeX writing assignment focusing on proofs and \textit{mathematical induction}. It was assigned by Professor Ken Takata and is due on 11/17/21.

\begin{definition}
Mathematical induction is a mathematical proof technique. It is essentially used to prove that a statement $P(n)$ holds for every natural number $n = 0, 1, 2, 3, ...$ (to be further explained in section 3.1).
\end{definition}

\section{Prove that $n^2 < 2^n$ for $n > 4$}

\subsection{Intro}
The principle of mathematical induction is to prove that $P(n)$ is true for all positive integers, $n$, where $P(n)$ is a propositional function. Two steps must be completed in mathematical induction:
\newline
\newline
- \textbf{Basis Step (or atomic step, base/atomic case):} Verify that $P(1)$ is true.
\newline
\newline
- \textbf{Inductive Step (or inductive case):} Show that the conditional statement $P(k) \rightarrow P(k+1)$ is true for all positive integers $k$.

\subsection{Base/Atomic Case}
Our base case will be $P(5)$ (or $P(4+1)$) since 5 is the first case that is greater than 4 which satisfies the conditions of the problem.

\begin{equation}
P(n, where \textrm{ } n>4) \equiv True \textrm{  means } n^2 < 2^n
\end{equation}
\newline
\newline
\begin{equation}
P(5) \textrm{  means } 5^2 < 2^5 \Longrightarrow 25 < 32
\end{equation}
\newline
\newline
This is true, and our base case is satisfied.

\subsection{Inductive Case}
Now we must show that the conditional statement $P(k) \rightarrow P(k+1)$ is true for all positive integers $k$.
\newline
\newline
\begin{equation}
\textrm{Assume  } P(k, where \textrm{ } k>4)\textrm{:   } k^2 < 2^k
\end{equation}
\newline
\newline
\begin{equation}
\textrm{Show  } P(k+1)\textrm{:   } (n+1)^2 < 2^{n+1} \rightarrow n^2 + 2n + 1 < 2 \times 2^n
\end{equation}
\newline
Assuming our integer is greater than 4, we can conclude that
\begin{equation}
2^{n+1} = n \times 2^n > 2 \times n^2 > (n+1)^2
\end{equation}
\newline
The left side of the equation is following the induction step by adding one to the base case. As for the right side
\begin{equation}
(n-1)^2 \ge 4^2 > 2
\end{equation}
\newline
Since $n \ge 5$, the inequality $(n-1)^2 > 2$ can be expanded using basic algebraic rules
\begin{equation}
n^2 - 2n - 1 > 2 
\end{equation}

\begin{equation}
n^2 - 2n - 1 > 0
\end{equation}

\begin{equation}
2n^2 - 2n - 1 > n^2
\end{equation}

\begin{equation}
2n^2 > n^2 + 2n + 1 
\end{equation}
\newline
Which can be expressed as
\begin{equation}
2n^2 > (n+1)^2 
\end{equation}
\newline
Which matches the second inequality in equation (5). Thus, we have proved the problem via mathematical induction. We can programmatically test this mathematical induction with the following code.



\section{Bonus - Code}

\begin{lstlisting}[language=Python, caption = This script tests if $n^2 > 2^n$ for a range of a given number greater than 4. A boolean for each iteration is passed to an list. If all booleans in the list are true we know our test has passed.]
def proof(n):
  if n <= 4:
    print("False. n <= 4 doesn't satisfy this equation")

  else:
    bool_array = []
    for i in range(5,n+1):
      bool_array.append(i**2 < 2**i)
    return all(bool_array)

>>> proof(10000)
True

 
\end{lstlisting}



\end{document}
