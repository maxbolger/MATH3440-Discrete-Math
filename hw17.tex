\documentclass{article}
\usepackage[utf8]{inputenc}
\usepackage{graphicx}
\usepackage{listings}
\usepackage[ruled, lined, linesnumbered, commentsnumbered, longend]{algorithm2e}
\usepackage{xcolor}
\newtheorem{definition}{Definition}

%New colors defined below
\definecolor{codegreen}{rgb}{0,0.6,0}
\definecolor{codegray}{rgb}{0.5,0.5,0.5}
\definecolor{codepurple}{rgb}{0.58,0,0.82}
\definecolor{backcolour}{rgb}{0.95,0.95,0.92}

\lstdefinestyle{mystyle}{
  backgroundcolor=\color{backcolour}, commentstyle=\color{codegreen},
  keywordstyle=\color{magenta},
  numberstyle=\tiny\color{codegray},
  stringstyle=\color{codepurple},
  basicstyle=\ttfamily\footnotesize,
  breakatwhitespace=false,         
  breaklines=true,                 
  captionpos=b,                    
  keepspaces=true,                 
  numbers=left,                    
  numbersep=5pt,                  
  showspaces=false,                
  showstringspaces=false,
  showtabs=false,                  
  tabsize=2
}

\lstset{style=mystyle}

\title{Homework 17}
\author{Max Bolger}
\date{October 2021}

\begin{document}

\maketitle

\section{Problem 1}

\textbf{Problem} - Find a way of calculating $1 + x + x^2 + x^3 + ... + x^n$ where $n$ is a non negative integer
\newline
\newline
\textbf{Solution} -  This solution assumes $1 + x + x^2 + x^3 + ... + x^n$ is equivalent to $x^0 + x^1 + x^2 + x^3 + ... + x^n$ as $x^0 = 1$ and $x^1 = x$

\begin{lstlisting}[language=Python, caption = This function uses a loop to solve the problem]
def func(x,n):
  sum = 0
  if n < 0:
    raise Exception('Sorry, n must be a non negative integer!')
  else:
    for i in range(n+1):
      sum+=(pow(x,i))

  return int(sum)
\end{lstlisting}
\newline
This can be written as $\frac{1 - x^{n+1}}{1-x}$. However, there is only one exception. If $x = 1$, then this equation is ignored and the answer is $n + 1$. This is a \textit{geometric series} since it has $n+1$ terms.

\begin{definition}[Geometric series]
A series whose terms form a geometric progression.
\end{definition}

\begin{lstlisting}[language=Python, caption = This function uses the mathematical formula to solve the problem]
def formula(x,n):
  sum = 0
  if n < 0:
    raise Exception('Sorry, n must be a non negative integer!')
  elif x == 1:
    sum = n+1
  else:
    sum = (1-(pow(x,n+1)))/(1-x)

  return int(sum)
 
\end{lstlisting}
\newline
We can write a third function that combines these two functions and checks if they are indeed equivalent

\pagebreak 

\begin{lstlisting}[language=Python, caption = A function that tests if my function and the mathematical formula to solve this problem are equal]
def test(x,n):
  sum_func = 0
  if n < 0:
    raise Exception('Sorry, n must be a non negative integer!')
  else:
    for i in range(n+1):
      sum_func+=(pow(x,i))

  if x == 1:
    sum_formula = n+1
  else:
    sum_formula = (1-(pow(x,n+1)))/(1-x)

  if sum_formula == sum_func:
    print('Equivalent')
  else:
    print('Not equivalent')

  return int(sum_formula), int(sum_func)
 
\end{lstlisting}
\newline
Lastly, we can test our function.

\begin{lstlisting}[language=Python, caption = Generating random numbers to test my function with]
>>>import random
>>> x = random.randint(0,10)
>>> n = random.randint(0,10)
>>> print(f'{x},{n}')
>>> test(x,n)
9,8
Equivalent
(48427561, 48427561)
\end{lstlisting}



\section{Problem 2}

\textbf{Problem} - \textit{True or False}: The \textit{power set} of the natural numbers is \underline{infinite but countable}.
\newline
\newline
\textbf{Solution} -  \textit{False}. The power set of the natural numbers is \underline{uncountably infinite}. \textit{Cantor's Theorem} states that for any set $A$ there is no onto function $A$ \rightarrow {\mathcal{P}(A)}. 
$ In this case, our set $A$ is all real numbers and the power set of all real numbers can't be countable as there is no onto function. Additionally, $ |{\mathcal{P}(s)}| $= 2^{|s|}$

\begin{definition}[Power set]
The set of all subsets of a set.
\end{definition}

\begin{definition}[Cantor's Theorem]
For any set A, the set of all subsets of A (the power set of A, denoted by ${\mathcal {P}(A)}$) has a strictly greater cardinality than A itself.
\end{definition}

\end{document}

